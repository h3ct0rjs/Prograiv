%%%%%%%%%%%%%%%%%%%%%%%%%%%%%%%%%%%%%%%%%
% Original author:
% Linux and Unix Users Group at Virginia Tech Wiki
% (https://vtluug.org/wiki/Example_LaTeX_chem_lab_report)
% Modified by: Hector F. Jimenez S, for my University Reports.
% License:
% CC BY-NC-SA 3.0 
%%%%%%%%%%%%%%%%%%%%%%%%%%%%%%%%%%%%%%%%%
%----------------------------------------
%	PACKAGES AND DOCUMENT CONFIGURATIONS
%---------------------------------------

\documentclass[paper=a4, fontsize=12pt]{article} 		% A4 paper and 11pt font size
\usepackage[T1]{fontenc} 								% Use 8-bit encoding that has 256 glyphs
%\usepackage{fourier}		 							% Use the Adobe Utopia font for the document 
\usepackage[spanish,english]{babel}						% Spanish Language, templates uses some sections in english.
\selectlanguage{spanish}								% main language.
\PassOptionsToPackage{spanish}{babel}
%\renewcommand{\figurename}{Figura}						% Force rename of figure.
%\renewcommand{\figurename}{Fig.}
\usepackage[figurename=Fig.]{caption}
\usepackage[utf8]{inputenc}								% tildes for spanish language.
\usepackage{amsmath,amsfonts,amsthm} 					% Math packages.
\usepackage{minted}										% For syntax highlighting.
\usepackage{float}										% Image will be in the same place as you want.!!! x-/
\usepackage{sectsty} 									% Allows customizing section commands
\allsectionsfont{\centering \normalfont\scshape}	   	% Make all sections centered, the default font and small caps
\usepackage{hyperref}
\hypersetup{											%Setups the false color and borders.
    colorlinks=false,
    pdfborder={0 0 0},
}
\newcommand\fnurl[2]{%									% set a simple and quick footnote command and include url.
\href{#2}{#1}\footnote{\url{#2}}%	
}
\usepackage{graphicx}									% Import easyly images.
\graphicspath{ {./images/} }							% Where to look for the images.
\DeclareGraphicsExtensions{.pdf,.png,.jpg}				% Graphics Extension to be used
\usepackage[notes,backend=biber]{biblatex-chicago}		% Bibliography and references.
\bibliography{biblio}									% bibliography filename.
\usepackage{fancyhdr} 									% Custom headers and footers
\pagestyle{fancyplain} 									% Makes all pages in the document conform to the custom headers and footers
\fancyhead{} 											% No page header
\fancyfoot[L]{} 										% Empty left footer
\fancyfoot[C]{} 										% Empty center footer
\fancyfoot[R]{\thepage} 								% Page numbering for right footer
\renewcommand{\headrulewidth}{0pt} 						% Remove header underlines
\renewcommand{\footrulewidth}{0pt} 						% Remove footer underlines
\setlength{\headheight}{13.6pt} 					    % Customize the height of the header
\numberwithin{equation}{section}						% Number equations within sections (i.e. 1.1, 1.2, 2.1, 2.2 instead of 1, 2, 3, 4)
%\numberwithin{figure}{section} 						% Number figures within sections (i.e. 1.1, 1.2, 2.1, 2.2 instead of 1, 2)
\numberwithin{table}{section} 							% Number tables within sections (i.e. 1.1, 1.2, 2.1, 2.2 instead of 1, 2, 3, 4)
\setlength\parindent{0pt} 								% Removes all indentation from paragraphs

\usepackage{listings}									% http://ctan.org/pkg/listings
\renewcommand{\lstlistingname}{Codigo}	

%\newcommand{\horrule}[1]{\rule{\linewidth}{#1}} 		% Create horizontal rule command with 1 argument of height
%%%%%%%%%%%%%%%%%%%%
%Title Section
%%%%%%%%%%%%%%%%%%%%%
\title{Problema de las 8 Reinas\\ 
Programación IV\\Laboratorio: 1} 						% Title
%\horrule{0.5pt} \\[0.4cm] 								% Thin top horizontal rule	Title rule
%\huge Assignment Title \\ 								% The assignment title
%\horrule{2pt} \\[0.5cm] 								% Thick bottom horizontal rule
\author{												% Authors begin.
Héctor F. \textsc{Jiménez Saldarriaga}\\
\texttt{hfjimenez@utp.edu.co} \\
\texttt{PGP KEY ID: 0xB05AD7B8}
} 												       % End of  Author name
\date{}    						                       % Date for the report, this will hide the \today.

\begin{document}
\maketitle                      			           % Insert the title, author and date
\begin{center}
\begin{tabular}{l r}								   % two column to
Fecha de Entrega: & \textbf{14} Febrero,2016 \\			
Profesor: & Ing. Ángel Augusto Agudelo Zapata
\end{tabular}
\end{center}
%%%%%%%%%%%	
% Let's start the document.
%%%%%%%%%%%	
\section{Restricciones de Objetivo}
Analizar, diseñar e implementar el problema de las 8 reinas en el lenguaje deseado.
\begin{itemize}
  \item El tamaño del tablero debe ser variable.
  \item Debe tomar tiempos para saber lo que se demora el algoritmo seleccionado.
  \item Debe mostrar el resultado en pantalla
  \item Debe generar una animación
\end{itemize}
$\\$
%%%%%%%%%%%	
% Theory Marc! 
%%%%%%%%%%%	
\section{Análisis de Problema}
%#https://www.inf.ethz.ch/personal/wirth/Articles/StepwiseRefinement.pdf
Para entender el problema a realizar se hizo la lectura del articulo provisto por el docente a cargo del autor \fnurl{Niklaus Wirth}{https://www.inf.ethz.ch/personal/wirth/Articles/StepwiseRefinement.pdf} en el articulo se plantea en como se debería de llevar acabo un diseño cuidadoso y modular cuando se intenta resolver un problema, el problema en contexto es el análisis del problema de las 8 reinas utilizando \emph{step refinement}, la idea es descomponer la solución planteada tanto como sea posible, estableciendo adecuadamente las $p$ posibles condiciones que acortan la solución. Para el problema de las 8 reinas \textbf{\textit{no existen}} soluciones analíticas y se deben explorar todas las posibles soluciones esto por obvias razones una solución por fuerza bruta seria costosa temporal y especialmente dado que tomaría $ \binom {64} {8}=\frac { 64! }{ 8!(64-8)! } =4,426,165,368$ combinaciones, por tal motivo es necesario reducir la búsqueda de soluciones a una que cumpla los requisitos y condiciones establecidas por nosotros; utilizando \textit{shortcuts} como lo menciona Niklaus tenemos en cuenta el movimiento ofensivo de la reina que puede atacar a $i$ elementos en su fila y columna correspondiente, así que solo debe de haber una y solo una reina por fila,columna $x,y$ del tablero, con esto podemos deducir que debemos permutar la $n$ cantidad de reinas que deseemos poner en el tablero, reduciendo esto a un número inferior de posibles 40,320 posibles ubicaciones, esto corresponde a $8!=40320$. Nuestra próxima condición será verificar los ataque en diagonales, que para para cada  permutación hallada, que representa un conjunto de posibles soluciones que satisfacen que ninguna reina se ataque a nivel horizontal y vertical; si hay almenos una solución dejaremos de buscar más sobre el conjunto de permutaciones. 

\section{Solución del Problema}
Teniendo en cuenta los hechos mencionados anteriormente asumiré que mi tablero máximo será un tablero de $n=10$ ya que debo acotar la solución de mi problema, además encontrar la cantidad de permutaciones en c++ no parece una tarea tan fácil con un n mayor a 10, como lo sería con el paquete \textbf{itertools} de \textbf{Python3}. Para resolver este problema he decidido utilizar \textbf{c++11}, para medir el tiempo utilizare la librería \fnurl{chrono}{http://www.cplusplus.com/reference/chrono/} y lo contrastare con el tiempo que me tire la utilidad \textbf{time} de Unix y presente en sistemas Gnu\/Linux, para realizar la animación he pensado utilizar ncurses(\textbf{sin confirmar por que tengo unos problemas para el render:\/}) en su ultima versión, dado que la idea es ver como se realiza el posicionamiento de las reinas en el tablero y no algo muy gráfico como sería hacerlo con allegro. 
\section{Anexos}
Los archivos correspondientes al proyecto de este laboratorio, junto con su readme y opciones soportadas pueden ser descargados del \fnurl{repositorio }{https://github.com/h3ct0rjs/ProgrammingIVassignments/tree/master/Lab1}
\end{document}
\end{document}